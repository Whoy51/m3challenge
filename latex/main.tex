\documentclass[letterpaper, oneside, 12pt]{report}

% include files
% Encoding packages
\usepackage[utf8]{inputenc}
\usepackage[T1]{fontenc}

% Margin/layout
\usepackage[letterpaper, margin=1in]{geometry}
\usepackage{fancyhdr}
\usepackage{lastpage}

% Misc
% \usepackage{blindtext}  % adds placeholder text


% header/footer style
\fancypagestyle{style}{%
  \fancyhf{}
  \fancyhead[LE,RO]{\slshape \rightmark}
  \fancyhead[LO,RE]{\slshape \leftmark}
  \fancyhead[L]{Team \#16656}
  \fancyhead[R]{\thepage}
  \fancyfoot[C]{--- Draft ---} % TODO: remove before final submission
  \renewcommand{\headrulewidth}{0.4pt}% Line at the header visible
  \renewcommand{\footrulewidth}{0.4pt}% Line at the footer visible
}

\pagestyle{style}


% guidelines and requirements
% https://m3challenge.siam.org/participate/rules-guidelines


\begin{document}
    \begin{abstract}
        Transportation plays a crucial part in today's modern society as it connects every country on every continent, and makes everyday commuting feasible. Essential products and services like raw materials, food, and trade are all dependent on effective means of transportation. Although governments do well in satisfying the demand for it, society often overlooks the devastating side effects that come with it such as its effectiveness in minimizing congestion, and most importantly its significant environmental impacts. For example, according to \textit{Global climate change and transportation infrastructure: lessons from the New York area} \cite{zimmerman1999global}, global warming is a counterintuitive byproduct that poses significant challenges for transportation infrastructure in coastal regions where many heavily-used transportation centers in New York are less than ten feet from sea level. Additionally, it brings out how exposure to rising temperatures affects the durability of pitch materials used for roads and bridges, and how, with the added stress from urban traffic, serious economic situations are bound to happen.

        So, what's being done about it? People are beginning to push for more renewable and cost effective forms of transportation such as electric bikes, and scooters. With that said, [data showing trends of e-bike usage over the years]. 
    \end{abstract}
    \section{Introduction}

    \section{Methods}
    \section{Results}

    \section{Conclusion}
    \bibliography{references}
\end{document}
